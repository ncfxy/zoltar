
% excerpt of TAIC PART

An important part of diagnosis and repair consists in
localizing faults.
Several tools for automated 
debugging and systems diagnosis implement an approach
to fault localization based on an analysis of the 
differences in program \emph{spectra} for \emph{passed} and \emph{failed}
runs. Passed runs are executions of a program that
are completed correctly, whereas failed runs are executions
in which an error was detected. A program spectrum is
an execution profile that indicates which parts of a 
program are active during a run. Fault localization entails
identifying the part of the program whose activity 
correlates most with the detection of errors. 

Spectrum-based fault localization (SFL) does not rely on
a model of the system under investigation. It can easily 
be integrated with existing testing procedures, and
because of the relatively small overhead with respect
to CPU time and memory requirements, it lends itself
well for application within resource-constrained 
environments. However, the efficiency of 
SFL comes at the cost of a limited
diagnostic accuracy. 
As an indication, in one of the 
experiments described in 
\cite{sfltaicpart}, 
on average
20\% of a program still needs to be inspected after the
diagnosis.

\section{Failures, Errors, and Faults}
\label{s:failuresErrorsFaults}
A \emph{failure} is an event that occurs when delivered service
deviates from correct service. An \emph{error} is a system
state that may cause a failure. A \emph{fault} is the cause of
an error in the system.
Since we focus on computer programs, faults are bugs in the program code, and failures occur
when the output for a given input deviates from the
specified output for that input.

\newpage

\lstinputlisting
	[captionpos=b, label=l:rationalSort, caption=A faulty C function for sorting rational numbers.]
	{sources/RationalSort_snippet.c}
	
To illustrate these concepts, consider the C 
function in Listing \ref{l:rationalSort}. 
It is meant to sort, using the 
bubble sort algorithm, a sequence of \verb|n| rational numbers
whose numerators and denominators are stored in the
parameters \verb|num| and \verb|den|, respectively. There is a fault
(bug) in the swapping code within the body of the if
statement: only the numerators of the rational 
numbers are swapped while the denominators are left in
their original order. In this case, a failure occurs
when \verb|RationalSort| changes the contents of its 
argument arrays in such a way that the result is not a
sorted version of the original. An error occurs after
the code inside the conditional statement is executed,
while \verb|den[j]| $\neq$ \verb|den[j+1]|. Such errors can be 
temporary, and do not automatically lead to failures. For
example, if we apply RationalSort to the sequence
$\langle \frac{4}{1}, \frac{2}{2}, \frac{0}{1} \rangle$,
an error occurs after the first two 
numerators are swapped. However, this error is "canceled" by
later swapping actions, and the sequence ends up being
sorted correctly.

Error detection is a prerequisite for SFL: we must know
that something is wrong before we can try to locate
the responsible fault. Failures constitute a 
rudimentary form of error detection, but many errors remain
latent and never lead to a failure. An example of a
technique that increases the number of errors that can
be detected is array bounds checking. Failure 
detection and array bounds checking are both examples of
generic error detection mechanisms, that can be 
applied without detailed knowledge of a program. Other
examples are the detection of null pointer handling,
malloc problems, and deadlock detection in concurrent
systems. Examples of program specific mechanisms are
precondition and postcondition checking, and the use
of assertions.

The Barinel toolset supports instrumenting for 
automatic error detection, which will be discussed in
Chapter \ref{c:AutomaticErrorDetection}.

\section{Program Spectra}
A program spectrum is a collection of data that
provides a specific view on the dynamic behavior of
software. This data is collected at run-time, and 
typically consist of a number of counters or 
flags for the
different parts of a program. Many different forms of
program spectra exist, some of which are supported by the Barinel toolset. 
In this example
we work with so-called block hit spectra.

A block hit spectrum contains a 
flag for every block
of code in a program, that indicates whether or not that
block was executed in a particular run. With a block of
code we mean a C language statement, where we do not
distinguish between the individual statements of a 
compound statement, but where we do distinguish between
the cases of a switch statement. As an illustration, we
have identified the blocks of code in Listing \ref{l:rationalSort}.


\section{Fault Localization}
The hit spectra of \emph{N} runs constitute a binary matrix,
whose columns correspond to \emph{M} different parts (blocks
in our case) of the program. The 
information in which runs an error was detected 
constitutes another column vector, the error vector. This
vector can be thought to represent a hypothetical part
of the program that is responsible for all observed 
errors. 
This is visualized in Figure \ref{fig:Ae}.
Fault localization essentially consists in 
identifying the part whose column vector resembles the error
vector most.

\begin{figure}[h!]
	\begin{center}
	\begin{tabular}{c c c}
							&					& error \\
							& $M$ components	& detection \\
			$N$ spectra 	& 
		$ \left[
		\begin{array}{c c c c}
			a_{11} & a_{12} & \ldots & a_{1M} \\
			a_{21} & a_{22} & \ldots & a_{2M} \\
			\vdots & \vdots & \ddots & \vdots \\
			a_{N1} & a_{N2} & \ldots & a_{NM} \\
		\end{array}
		\right] $
		&
		$ \left[
		\begin{array}{c}
			e_1 \\
			e_2 \\
			\vdots \\
			e_N \\
		\end{array}
		\right] $
		\\
	\end{tabular}
	\caption{Input to SFL}
	\label{fig:Ae}
	\end{center}
\end{figure}

In the field of data clustering, resemblances between
vectors of binary, nominally scaled data, such as the
columns in our matrix of program spectra, are 
quantified by means of similarity coefficients.
By default, the Barinel toolset uses the Ochiai coefficient
$s_O$, used in the molecular biology domain,
since this coefficient performed best in experiments
\cite{sfltaicpart}:\\
\begin{center}
$s_O(j) = \frac{n_{11}(j)}{\sqrt{(n_{11}(j)+n_{01}(j))*(n_{11}(j)+n_{10}(j))}}$\\
\end{center}
where $n_{pq}(j)$ indicates the number of runs in which block \emph{j} 
has been touched during the execution, denoted by $p \in \{0,1\}$,
and where the runs are failed or passed, indicated by $q \in \{0,1\}$:
\begin{center}
\begin{tabular}{c|c|c}
$n$ & touched & run \\
\hline
$n_{00}$ & no & passed \\
$n_{10}$ & yes & passed \\
$n_{01}$ & no & failed \\
$n_{11}$ & yes & failed \\
\end{tabular}
\end{center}

Under the assumption that a high similarity to the
error vector indicates a high probability that the corresponding parts of the software cause the detected
errors, the calculated similarity coefficients rank the
parts of the program with respect to their likelihood of
containing the faults.

To illustrate the approach, suppose that we 
apply the \verb|RationalSort| function to the input sequences
$I_1$, \ldots ,$I_6$ (see below). The block hit spectra for these
runs are as follows ('1' denotes a hit), where block 5
corresponds to the body of the \verb|RationalGT| function,
which has not been shown in Listing \ref{l:rationalSort}.
\begin{center}
\begin{tabular}{l|c c c c c|c}
\multicolumn{1}{c}{} & \multicolumn{5}{c}{block} & \\
input & 1 & 2 & 3 & 4 & 5 & error \\
\hline
$I_1 = \langle \rangle$ 
& 1 & 0 & 0 & 0 & 0 & 0 \\
$I_2 = \langle \frac{1}{4} \rangle$ 
& 1 & 1 & 0 & 0 & 0 & 0 \\
$I_3 = \langle \frac{2}{1}, \frac{1}{1} \rangle$ 
& 1 & 1 & 1 & 1 & 1 & 0 \\
$I_4 = \langle \frac{4}{1}, \frac{2}{2}, \frac{0}{1} \rangle$ 
& 1 & 1 & 1 & 1 & 1 & 0 \\
$I_5 = \langle \frac{3}{1}, \frac{2}{2}, \frac{4}{3}, \frac{1}{4} \rangle$ 
& 1 & 1 & 1 & 1 & 1 & 1 \\
$I_6 = \langle \frac{1}{4}, \frac{1}{3}, \frac{1}{2}, \frac{1}{1} \rangle$ 
& 1 & 1 & 1 & 0 & 1 & 0 \\
\hline
\multicolumn{1}{c|}{$s_O$} & $.40$ & $.44$ & $.50$ & $.58$ & $.50$ & \\
\end{tabular}
\end{center}

$I_1$, $I_2$, and $I_6$ are already sorted, and lead to passed
runs. $I_3$ is not sorted, but the denominators in this
sequence happen to be equal, hence no error occurs. $I_4$
is the example from Section \ref{s:failuresErrorsFaults}: 
an error occurs during
its execution, but goes undetected. For $I_5$ the program
fails, since the calculated result is 
$\langle \frac{1}{1}, \frac{2}{2}, \frac{4}{3}, \frac{3}{4} \rangle$
instead of 
$\langle \frac{1}{4}, \frac{2}{2}, \frac{4}{3}, \frac{3}{1} \rangle$
, which is a clear indication that an error
has occurred. For this data, the calculated similarity
coefficient $s_O$ (correctly)
identifies block 4 as the most likely location of the fault.
In general, however, the faulty component(s) may be outranked
by other components, entailing non-zero search effort by the users.
Research has shown that for small programs ($O(100)$ lines) 
$5 - 20 \%$ of the code remains to be inspected by the user \cite{ZAGGECBS:07}.
However, for large programs this fraction drops to less than a percent \cite{ijcai09},
making SFL an interesting debugging aid.

