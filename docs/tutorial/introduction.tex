
Debugging is an important part of software development.
Several debugging tools exist which are based on stepping through the execution of the program.
These tools require the user to have knowledge about the code that is executed and 
about the inner workings and dependencies in the program 
which can be very difficult for one person to comprehend.

In practice, a large test set usually exists against which the program under development is tested.
Some tests will fail, some will pass, and this information can indicate what is going wrong
when unexpected behavior occurs.
However, in many cases, especially for large programs, 
a tester needs extensive knowledge about the program to be able to 
map a certain output or behavior to a point in the source code.

% With different testing and debugging methods available, 
% still there are times at which it is uncertain which locations of the source code should be investigated.
% These moments of undecidability can cause a great delay in software development
% and can limit the completeness and the stability of the product,
% given the relatively short time-to-market these days.

A solution to these issues would be a black box method 
which takes a program and available test cases 
and returns the most probable location of the fault
in case a number of these tests fail.
% In this tutorial a set of tools is introduced and explained
% which is able to do just that.
In this tutorial a toolset is introduced that implements
a technique called Spectrum-based Fault Localization (SFL \cite{sfltaicpart}).
SFL is based on instrumenting a program and keeping track of executed parts of the code
after which the spectrum-based fault localization technique is applied to 
return a list of source code locations ordered by the likelihood of it containing the fault.
Furthermore, the tool set enables a program to be trained with expected behavior
and to automatically detect an error if it encounters unexpected behavior.
The fact that no knowledge is needed of the program to acquire possible fault locations
would make this set of tools a useful extension to 
currently applied methods of testing and debugging.

The toolset is called \textsc{Zoltar}
%\footnote{\textsc{Barinel} stands
%for Bayesian AppRoach to dIagnosing iNtErmittent fauLts. A barinel is a type
%of caravel used by the Portuguese sailors during their discoveries.}
and is the result of research done at TUD in the context
of the TRADER project~\cite{trader},
involving several Dutch universities, Philips Tass, IMEC, and NXP
Semiconductors and is conducted under the
responsibility of the Embedded Systems Institute (ESI) in Eindhoven, the
Netherlands. The main goal of the
TRADER project was to develop methods and tools for ensuring reliability
of consumer electronic
products, minimizing product failures that are exposed to the end user.

This tutorial is organized as follows.
Chapter \ref{c:SFL} gives background information on the spectrum-based fault localization
technique which is adopted by the Zoltar toolset.
In Chapter \ref{c:Installation} the installation of the tools is explained
and the different tools of the package are discussed.
In Chapter \ref{c:ExampleProgramAnalysis} an example program is discussed
which is then instrumented and analyzed using the tools.
It will show the basics of the tools and enables the reader to perform
a quick program analysis.
Chapter \ref{c:ProgramSpectrumGeneration} gives some details of 
the instrumentation of program points which together will create program spectra.
Chapter \ref{c:AutomaticErrorDetection} contains information on 
the instrumentation, training and testing of a program in order to 
support automatic error detection.
The process of instrumentation is generalized in Chapter \ref{c:AnalyzingLargePrograms}
to be able to efficiently analyze large projects consisting of multiple source files.
Chapter \ref{c:BatchExecution} contains information on options in the tools to aid in
batch execution of tests.
Finally, the appendices contain detailed information of techniques and developers information
for the tool set.
